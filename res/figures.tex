\def\bfig#1#2{\expandafter\gdef\csname fig-#1\endcsname{\begin{figure}[ht]#2\label{#1}\end{figure}}}
\def\fref#1{\csname fig-#1\endcsname\Cref{#1}}

\bfig{fig:n3t553_start_sites}{
    \centering
    \begin{subfigure}{3cm}
        \includegraphics[width=\textwidth]{n3_t553_start_b-cropped.pdf}
        \caption{bond}
    \end{subfigure}
    \hspace{1cm}
    \begin{subfigure}{3cm}
        \includegraphics[width=\textwidth]{n3_t553_start_h-cropped.pdf}
        \caption{hollow}
    \end{subfigure}
    \hspace{1cm}
    \begin{subfigure}{3cm}
        \includegraphics[width=\textwidth]{n3_t553_start_z-cropped.pdf}
        \caption{zigzag}
    \end{subfigure}
    \caption{Starting sites of the bond, hollow and zigzag geometries, respectively.}
}

\bfig{fig:n3t553_start_center}{
    \centering
    \begin{subfigure}{3cm}
        \includegraphics[width=\textwidth]{n3_t553_start_c-cropped.pdf}
    \end{subfigure}
    \hspace{1cm}
    \begin{subfigure}{3cm}
        \includegraphics[width=\textwidth]{n3_t553_start_c_side-cropped.pdf}
    \end{subfigure}
    \caption{Central starting geometry (c). Note that the nanotube model has dangling hydrogen atoms at the two ends which are not shown here.}
}

\bfig{fig:n3t553_geom_parms}{
    \centering
    \begin{subfigure}{4.7cm}
        \includegraphics[width=\textwidth]{n3_t553_dist-cropped.pdf}
    \end{subfigure}
    \hspace{1cm}
    \begin{subfigure}{4cm}
        \includegraphics[width=\textwidth]{n3_t553_angle-cropped.pdf}
    \end{subfigure}
    \caption{Geometrical parameters.}
}

\bfig{fig:n3t55x_ab-initio_eint}{
    \centering
    \input{n3_t55x_ab-initio_eint}
    \caption{Interaction energy at \emph{ab-initio} level as a function of tube length.}
}

\bfig{fig:n3t55x_classical_eint}{
    \centering
    % GNUPLOT: LaTeX picture with Postscript
\begingroup
  \makeatletter
  \providecommand\color[2][]{%
    \GenericError{(gnuplot) \space\space\space\@spaces}{%
      Package color not loaded in conjunction with
      terminal option `colourtext'%
    }{See the gnuplot documentation for explanation.%
    }{Either use 'blacktext' in gnuplot or load the package
      color.sty in LaTeX.}%
    \renewcommand\color[2][]{}%
  }%
  \providecommand\includegraphics[2][]{%
    \GenericError{(gnuplot) \space\space\space\@spaces}{%
      Package graphicx or graphics not loaded%
    }{See the gnuplot documentation for explanation.%
    }{The gnuplot epslatex terminal needs graphicx.sty or graphics.sty.}%
    \renewcommand\includegraphics[2][]{}%
  }%
  \providecommand\rotatebox[2]{#2}%
  \@ifundefined{ifGPcolor}{%
    \newif\ifGPcolor
    \GPcolortrue
  }{}%
  \@ifundefined{ifGPblacktext}{%
    \newif\ifGPblacktext
    \GPblacktexttrue
  }{}%
  % define a \g@addto@macro without @ in the name:
  \let\gplgaddtomacro\g@addto@macro
  % define empty templates for all commands taking text:
  \gdef\gplbacktext{}%
  \gdef\gplfronttext{}%
  \makeatother
  \ifGPblacktext
    % no textcolor at all
    \def\colorrgb#1{}%
    \def\colorgray#1{}%
  \else
    % gray or color?
    \ifGPcolor
      \def\colorrgb#1{\color[rgb]{#1}}%
      \def\colorgray#1{\color[gray]{#1}}%
      \expandafter\def\csname LTw\endcsname{\color{white}}%
      \expandafter\def\csname LTb\endcsname{\color{black}}%
      \expandafter\def\csname LTa\endcsname{\color{black}}%
      \expandafter\def\csname LT0\endcsname{\color[rgb]{1,0,0}}%
      \expandafter\def\csname LT1\endcsname{\color[rgb]{0,1,0}}%
      \expandafter\def\csname LT2\endcsname{\color[rgb]{0,0,1}}%
      \expandafter\def\csname LT3\endcsname{\color[rgb]{1,0,1}}%
      \expandafter\def\csname LT4\endcsname{\color[rgb]{0,1,1}}%
      \expandafter\def\csname LT5\endcsname{\color[rgb]{1,1,0}}%
      \expandafter\def\csname LT6\endcsname{\color[rgb]{0,0,0}}%
      \expandafter\def\csname LT7\endcsname{\color[rgb]{1,0.3,0}}%
      \expandafter\def\csname LT8\endcsname{\color[rgb]{0.5,0.5,0.5}}%
    \else
      % gray
      \def\colorrgb#1{\color{black}}%
      \def\colorgray#1{\color[gray]{#1}}%
      \expandafter\def\csname LTw\endcsname{\color{white}}%
      \expandafter\def\csname LTb\endcsname{\color{black}}%
      \expandafter\def\csname LTa\endcsname{\color{black}}%
      \expandafter\def\csname LT0\endcsname{\color{black}}%
      \expandafter\def\csname LT1\endcsname{\color{black}}%
      \expandafter\def\csname LT2\endcsname{\color{black}}%
      \expandafter\def\csname LT3\endcsname{\color{black}}%
      \expandafter\def\csname LT4\endcsname{\color{black}}%
      \expandafter\def\csname LT5\endcsname{\color{black}}%
      \expandafter\def\csname LT6\endcsname{\color{black}}%
      \expandafter\def\csname LT7\endcsname{\color{black}}%
      \expandafter\def\csname LT8\endcsname{\color{black}}%
    \fi
  \fi
    \setlength{\unitlength}{0.0500bp}%
    \ifx\gptboxheight\undefined%
      \newlength{\gptboxheight}%
      \newlength{\gptboxwidth}%
      \newsavebox{\gptboxtext}%
    \fi%
    \setlength{\fboxrule}{0.5pt}%
    \setlength{\fboxsep}{1pt}%
\begin{picture}(6800.00,6800.00)%
    \gplgaddtomacro\gplbacktext{%
      \csname LTb\endcsname%%
      \put(645,852){\makebox(0,0)[r]{\strut{}$-60$}}%
      \csname LTb\endcsname%%
      \put(645,1365){\makebox(0,0)[r]{\strut{}$-50$}}%
      \csname LTb\endcsname%%
      \put(645,1878){\makebox(0,0)[r]{\strut{}$-40$}}%
      \csname LTb\endcsname%%
      \put(645,2391){\makebox(0,0)[r]{\strut{}$-30$}}%
      \csname LTb\endcsname%%
      \put(645,2905){\makebox(0,0)[r]{\strut{}$-20$}}%
      \csname LTb\endcsname%%
      \put(645,3418){\makebox(0,0)[r]{\strut{}$-10$}}%
      \csname LTb\endcsname%%
      \put(645,3931){\makebox(0,0)[r]{\strut{}$0$}}%
      \csname LTb\endcsname%%
      \put(645,4445){\makebox(0,0)[r]{\strut{}$10$}}%
      \csname LTb\endcsname%%
      \put(645,4958){\makebox(0,0)[r]{\strut{}$20$}}%
      \csname LTb\endcsname%%
      \put(645,5471){\makebox(0,0)[r]{\strut{}$30$}}%
      \csname LTb\endcsname%%
      \put(645,5984){\makebox(0,0)[r]{\strut{}$40$}}%
      \csname LTb\endcsname%%
      \put(1189,409){\makebox(0,0){\strut{}$1$}}%
      \csname LTb\endcsname%%
      \put(2073,409){\makebox(0,0){\strut{}$3$}}%
      \csname LTb\endcsname%%
      \put(2957,409){\makebox(0,0){\strut{}$5$}}%
      \csname LTb\endcsname%%
      \put(3841,409){\makebox(0,0){\strut{}$7$}}%
      \csname LTb\endcsname%%
      \put(4725,409){\makebox(0,0){\strut{}$9$}}%
      \csname LTb\endcsname%%
      \put(5609,409){\makebox(0,0){\strut{}$11$}}%
      \csname LTb\endcsname%%
      \put(6493,409){\makebox(0,0){\strut{}$13$}}%
    }%
    \gplgaddtomacro\gplfronttext{%
      \csname LTb\endcsname%%
      \put(153,3418){\rotatebox{-270}{\makebox(0,0){\strut{}Interaction energy (kcal/mol)}}}%
      \csname LTb\endcsname%%
      \put(3620,130){\makebox(0,0){\strut{}\# units}}%
      \csname LTb\endcsname%%
      \put(1979,6633){\makebox(0,0)[r]{\strut{}\emph{ab-initio}}}%
      \csname LTb\endcsname%%
      \put(1979,6447){\makebox(0,0)[r]{\strut{}Induction}}%
      \csname LTb\endcsname%%
      \put(4093,6633){\makebox(0,0)[r]{\strut{}Electrostatic}}%
      \csname LTb\endcsname%%
      \put(4093,6447){\makebox(0,0)[r]{\strut{}Dispersion}}%
      \csname LTb\endcsname%%
      \put(6207,6633){\makebox(0,0)[r]{\strut{}Total classical}}%
    }%
    \gplbacktext
    \put(0,0){\includegraphics{n3_t55x_classical_eint}}%
    \gplfronttext
  \end{picture}%
\endgroup

    \caption{Interaction energy using the classical potential.}
}

\bfig{fig:n3t55x_comp_eint}{
    \centering
    \input{n3_t55x_comp_eint}
    \caption{Comparison of classical and quantum interaction energies.}
}
