\documentclass{article}

\usepackage{preamble}

\begin{document}

\section{Computational Details}

\subsection{Ab-initio}
In this work, the $(5,5)$ armchair carbon nanotubes have been treated as finite-size systems, with the addition of hydrogen atoms at both ends in order to complete the valence shell.
The nanotubes geometries have been optimized at DFT level of theory, employing the B97D3 exchange-correlation functional\cite{Grimme2010} in conjunction with the correlation consistent cc-pvtz gaussian basis set\cite{DunningJr1989}.
All the optimized structures retained their high symmetry, formally belonging to the $D_{5d}$ molecular point group.
Since nanotubes of different lengths have been considered throughout the work, we introduce the notation $\Lambda - \text{CNT}(5,5)$ to label a nanotube composed by $\Lambda$ [$10$]cyclophenacene units (cf. \Cref{}).
The C--C bonds parallel and perpendicular to the principal axis show the alternation pattern typical of these systems when treated as finite clusters\cite{Zhou2004,Galano2006}.\\
The geometry of the N$_3^-$ azide anion has been optimized at DFT level of theory too, with the same functional, but in conjunction with the augmented version of the Dunning basis set, the aug-cc-pvtz in order to better describe the more diffuse electron density.
The N--N bond length measures $1.1874$ \AA and is in excellent agreement with the experimental value obtained in gas phase of $1.1884$ \AA\cite{Polak1987}.
The geometry optimizations of the fragments alone have been performed with the Gaussian 09 software package, revision d01\cite{g09}, using the default convergence thresholds and the ultrafine grid for the functional integral calculation.

\subsubsection{N$_3^-$ inside CNT$(5,5)$}
It has been recently shown\cite{Battaglia2017a} that the interaction between the two fragments does not change significantly their structure.
Therefore, the optimal adsorption distance between the azide anion and the inner wall of the carbon nanotube has been obtained by relaxing all internal coordinates connecting the two fragments, while keeping frozen those strictly belonging to the monomers only.
To ensure that the relaxation process did not remain trapped in a local minimum and without knowing a priori the optimal adsorption site of the N$_3^-$ molecule, the optimization procedure has been started from three different geometries corresponding to different possible adsorption sites, depicted in \Cref{}.
The relaxation has been performed at density functional theory level using the B97D3 functional with the cc-pvdz basis set on the carbon nanotube and the aug-cc-pvtz on the N$_3^-$. The RI approximation has been used to compute the Coulomb part, with the universal fitting basis set def2/j\cite{weige}



\subsection{Molecular Dynamics}
The Improved Lennard-Jones (ILJ) potential is given by
%
\begin{equation}
    V_{ILJ}(r) = \epsilon \left[ \frac{m}{n(r)-m} \cdot \Big( \frac{r_0}{r} \Big)^{n(r)} -
                                 \frac{n(r)}{n(r)-m} \cdot \Big( \frac{r_0}{r} \Big)^{m} \right] \, ,
\end{equation}
%
where
%
\begin{equation}
    n(r) = \beta + 4.0 \cdot \Big( \frac{r}{r_0} \Big)^2
\end{equation}
%
and $r = R_{i,j}$ is the distance between carbon atom $i$ and nitrogen atom $j$. The parameters $r_0$, $\epsilon$ and $\beta$ are set according to the atoms interacting, while $m$ is set according the to the partial charges of the interacting species.
The induction potential term for a $\text{N}_3^-$ anion interacting with carbon atoms is given by (in meV)
%
\begin{equation}
    V_{ind}(R_{i,1},R_{i,2},R_{i,3}) = - 7200 \cdot \alpha_c \cdot \frac{n_2(R_{i,2})}{n_2(R_{i,2}) - 4}
                        \left[ \frac{q_1}{R_{i,1}^2} + \frac{q_2}{R_{i,2}^2} + \frac{q_3}{R_{i,3}^2} \right]^2 \, ,
\end{equation}
%
where $R_{i,1}$, $R_{i,2}$ and $R_{i,3}$ are the distances between the carbon atom $i$ and the three nitrogen atoms, $q_1$, $q_2$ and $q_3$ are the partial charges on the nitrogen atoms, $\alpha_c$ is the polarizability of the carbon atoms and $n_2(R_{i,2})$ is equal to $n(r)$ using $r_0$ according to the interaction between $C-N_2$.\\
The interaction energy between a trinitrogen anion confined inside a carbon nanotube is given by
%
\begin{align}
    E_{int} = \sum_{i=1}^{N_C} V_{ILJ}^{C-N_{1,3}}(R_{i,1}) + V_{ILJ}^{C-N_{2}}(R_{i,2}) V_{ILJ}^{C-N_{1,3}}(R_{i,3}) + V_{ind}(R_{i,1},R_{i,2},R_{i,3}) \, ,
\end{align}
%
where $N_C$ corresponds to the total number of carbon atoms, the ILJ potentials $V_{ILJ}^{C-N_{1,3}}$ and $V_{ILJ}^{C-N_{2}}$ are defined according to the parameters given in \Cref{tab:ILJ_parms} and the polarizability $\alpha_c$ appearing in $V_{ind}$ is set to $1.2$, the partial charges as $q_1 = q_3 = -0.56$ and $q_2 = 0.12$ and finally the term $\tfrac{n_2}{n_2 - 4}$ is simplified to $1.0$ for the time being.
%
\begin{table}[h!]
    \centering
    \begin{tabular}{l|cccc}
        atom types      & $\epsilon$  & $r_0$     & $\beta$ & $m$ \\
        \midrule
        C-\ce{N_{1,3}}  & 5.205       & 3.994     & 6--9    & 6.0 \\
        C-\ce{N_{2}}    & 3.536       & 3.828     & 6--9    & 6.0 \\
    \end{tabular}
    \caption{ILJ parameters for the interaction of $\text{N}_3^-$ confined inside a carbon nanotube. Different values of $\beta$ are tested.}
    \label{tab:ILJ_parms}
\end{table}
%
In case of large polarization (large partial charges), the electrostatic interaction between interaction centers has to be taken into account. The fact that we have a charged species ($\text{N}_3^-$), enhances the polarization and it really needs to be taken into account.\\
\\
Since the aim is to fit the potential to the max interaction energy only (i.e. to fit the well depth $\epsilon$), a very accurate value for this interaction energy is required. In order to do so, an elaborate scheme is required, which involves several steps.\\
%
\begin{enumerate}
    \item Geometry relaxation of the complex system and the monomers. The nanotube length is only 3 units, otherwise there are too many basis functions
    \begin{enumerate}
        \item Using one pure functional among PBE, B97 and TPSS
        \item Using Grimme dispersion with Becke and Johnson damping (D3BJ)
        \item Using the cc-pvtz basis set on the CNT and the aug-cc-pvtz on the nitrogen (diffuse functions also on the tube lead to linear dependence)
        \item Now testing density fitting in Gaussian, if much faster, we use also that for the pure functional with the universal def2/j basis set
        \item Using one hybrid functional among B3LYP, APFD, what else?
    \end{enumerate}
    \item All the geometry calculations are performed in Gaussian
    \item To obtain accurate interaction energies, there is a total of 7 single point calculations to be performed
    \begin{enumerate}
        \item Complex at the complex geometry in the full basis
        \item First monomer at the complex geometry in the full basis
        \item First monomer at the complex geometry in the monomer basis
        \item First monomer at the monomer geometry in the monomer basis
        \item Second monomer at the complex geometry in the full basis
        \item Second monomer at the complex geometry in the monomer basis
        \item Second monomer at the monomer geometry in the monomer basis
    \end{enumerate}
    \item The series of 7 calculations just listed has to be done for two basis sets, to be able to extrapolate the results at CBS limit
    \item The extrapolation schemes are different for SCF and correlated calculations, therefore they can be done separately
    \item For the SCF extrapolation, we can compute in Gaussian using triple and quadruple zeta basis sets
    \item For the correlated extrapolation, we can compute in ORCA using double and triple zeta basis sets (which will give us also dz scf points)
    \item The SCF in this case does not use the RI approximation for both Gaussian and ORCA
    \item The correlated method is the DLPNO-CCSD(T)
    \item Partial charges have to be obtained and there are many ways to do it. First of all, we need to decide which density to use
    \begin{enumerate}
        \item The density (relaxed, unrelaxed, linearized?) of the DLPNO-CCSD(T) method (I do not know how much expensive it is)
        \item Find a DFT functional in which we can trust (how?) and use its density
        \item Use the density of DLPNO-SCS-MP2 if it reproduces well the CCSD(T) one. But the same problem as above (relaxed, unrelaxed, linearized? Expensive?)
    \end{enumerate}
    \item Once we know which density we want to use, calculate the charges
    \begin{enumerate}
        \item NPA charges from NBO analysis
        \item MESP derived charges
        \item Many other possibilities...
    \end{enumerate}
    \item Calculate multipole moments with the charges obtained and compare them with the moments obtained from the calculation
\end{enumerate}
%
The above computational scheme will provide a very accurate interaction energy and can be ideally applied to different systems. The major pitfall of the above procedure, is the very first step, i.e. the geometry optimization.
%
\begin{enumerate}
    \item The functional used for the geometry optimization might give wrong conformations
    \item One should optimize the geometry using a wf method, but if it is too expensive we cannot
    \item Moreover there are a plethora of minima in the PES, several starting geometries have to be relaxed and compared with each other
\end{enumerate}
%
\medskip
With the interaction energy and the partial charges, we can tune the ILJ potential such that it reproduces the interaction energy obtained with the above scheme.\\
\newline
There is a large amount of arbitrariness in this procedure to obtain the ILJ potential.
The partial charges calculated are in principle arbitrary and considering them explicitly enhances the complexity of the problem. In particular, the charges will certainly vary depending on the length of the nanotube. This means that if we would like to use longer CNTs in the MD simulation, then the tuned ILJ potential which includes explicitly the charges might not be valid anymore.
I believe that therefore we cannot pursue this approach.\\
A possible solution would be to fit the ILJ potential to a series of calculations of the complex system at different conformations. In this way, the electrostatic effects would be implicitly taken care of in the fitting procedure. Nonetheless, for such a fit, the above procedure is certainly to expensive and one should come up with a reasonable approximation.\\
Another possibility is to neglect completely the electrostatic term in the potential and tune it just according dispersion and induction forces. This is basically the same as the above solution, but without fitting the entire potential and only relying on the single interaction energy value.\\
A last idea is to assume that in the limit of a very long nanotube, the polarization far away from the nanotube ends is minimal and has to be taken care only locally around the azide anion. This would mean that the partial charges are assigned only locally around the azide anion and they would need to be approximated in some way. One such way is to compute them for a nanotube of increasing length and observe how they change. If the change is minimal, they can be safely assumed to remain so for any length, if this is not the case, then probably this idea is not feasible.

%
% \begin{table}
    % \centering
    % \begin{tabular}{c|ccc}
        % Method      & $r_{NN}$  & Q$_{1,3}$ & Q$_2$ \\
        % \midrule
        % Exp.\cite{Polak1987} & $1.1884$  & -         & -
    % \end{tabular}
% \end{table}
%

%
\begin{table}
    \centering
    \begin{tabular}{r ccc}
        \toprule
        $\Lambda$-CNT & 3-CNT     & 5-CNT     & 7-CNT    \\
        occ./virt.    & 260/2800  & 380/4000  & 500/5200 \\
        \midrule
        RI-B97D3      & $-30.80$  & $-38.29$  & $-42.58$ \\
        \textit{Time} & 27m       & 50m       & 1h20m    \\
        \vspace{1mm} \\
        RI-SCS-MP2    & $-34.56$  & $-43.36$  & $-$      \\
        \textit{Time} & 8h        & 36h (8cpus) & $-$    \\
        \bottomrule
    \end{tabular}
    \caption{CP-corrected unrelaxed interaction energies of N$_3^-$ confined inside a $(5,5)$ SWCNT of increasing length. The azide anion is placed perfectly in the center of the nanotube, parallel to the principal axis. The basis set use is the \textbf{cc-pvtz} with the addition of diffuse functions (aug-) on the anion. All energies are given in kcal/mol.}
\label{tab:eint_cnt55x_cc-pvtz}
\end{table}
%
\begin{table}
    \centering
    \begin{tabular}{r cccc}
        \toprule
        $\Lambda$-CNT & 3-CNT     & 5-CNT     & 7-CNT      & 9-CNT    \\
        occ./virt.    & 260/1400  & 380/1900  & 500/2500   & 620/3000 \\
        \midrule
        RI-B97D3      & $-26.26$  & $-33.97$  & $-38.50$   & $-41.85$ \\
        \textit{Time} & 9m        & 20m       & 33m        & 45m \\
        \vspace{1mm} \\
        RI-SCS-MP2    & $-29.77$  & $-38.87$  & $-43.58$   & $-47.07$ \\
        \textit{Time} & 1h40m     & 6h        & 10h        & 24h \\
        \vspace{1mm} \\
        DLPNO-CCSD    & $-$       & $-$       & $-$        & $-$ \\
        \textit{Time} & est. 36h  & $-$       & $-$        & $-$ \\
        \bottomrule
    \end{tabular}
    \caption{CP-corrected unrelaxed interaction energies of N$_3^-$ confined inside a $(5,5)$ SWCNT of increasing length. The azide anion is placed perfectly in the center of the nanotube, parallel to the principal axis. The basis set use is the \textbf{cc-pvdz} on the nanotube and the aug-cc-pvtz on the anion. All energies are given in kcal/mol.}
\label{tab:eint_cnt55x_cc-pvdz}
\end{table}
%



% \Cref{tab:eint_geo_1_ind_1} shows the interaction energies computed using the above formula and different values of $\beta$ for a representative geometry.
% %
% \begin{table}
%     \centering
%     \begin{tabular}{l|rrrr}
%         $E_{int}$            & $\beta = 6.0$ & $\beta = 7.0$ & $\beta = 8.0$ & $\beta = 9.0$ \\
%         \midrule
%         $V_{ILJ}$ [meV]      &  -569.17      &  -517.63      &  -462.66      &  -403.66      \\
%         $V_{ILJ}$ [kcal/mol] &   -13.12      &   -11.94      &   -10.67      &    -9.31      \\
%         $V_{ind}$ [meV]      & -2493.52      & -2493.52      & -2493.52      & -2493.52      \\
%         $V_{ind}$ [kcal/mol] &   -57.50      &   -57.50      &   -57.50      &   -57.50      \\
%         \midrule
%         $E_{int}$ [meV]      & -3062.68      & -3011.15      & -2956.18      & -2897.18      \\
%         $E_{int}$ [kcal/mol] &   -70.63      &   -69.44      &   -68.17      &   -66.81
%     \end{tabular}
%     \caption{Interaction energies computed using ILJ and the induction term for different $\beta$ values. The term $\tfrac{n_2}{n_2 - 4}$ in the induction potential was kept fixed equal one. The reference MP2 value is -32.31 kcal/mol (-1401.13 meV).}
%     \label{tab:eint_geo_1_ind_1}
% \end{table}
% %
% A second series of test has been done by resetting the term $\tfrac{n_2}{n_2 - 4}$, which depends on $\beta$ and therefore will change accordingly.
% \Cref{tab:eint_geo_1_ind_beta} shows the results in this case.
% %
% \begin{table}
%     \centering
%     \begin{tabular}{l|rrrr}
%         $E_{int}$            & $\beta = 6.0$ & $\beta = 7.0$ & $\beta = 8.0$ & $\beta = 9.0$ \\
%         \midrule
%         $V_{ILJ}$ [meV]      &  -569.17      &  -517.63      &  -462.66      &  -403.66      \\
%         $V_{ILJ}$ [kcal/mol] &   -13.12      &   -11.94      &   -10.67      &    -9.31      \\
%         $V_{ind}$ [meV]      & -4033.24      & -3818.14      & -3657.00      & -3531.54      \\
%         $V_{ind}$ [kcal/mol] &   -93.01      &   -88.05      &   -84.33      &   -81.44      \\
%         \midrule
%         $E_{int}$ [meV]      & -4602.40      & -4335.77      & -4119.67      & -3935.20      \\
%         $E_{int}$ [kcal/mol] &  -106.13      &   -99.98      &   -95.00      &   -90.75
%     \end{tabular}
%     \caption{Interaction energies computed using ILJ and the induction term for different $\beta$ values. The term $\tfrac{n_2}{n_2 - 4}$ in the induction potential was let free to vary. The reference MP2 value is -32.31 kcal/mol (-1401.13 meV).}
%     \label{tab:eint_geo_1_ind_beta}
% \end{table}
% %
% A third series of calculations has been performed using NBO charges calculated on the MP2 density; namely $q_1 = q_3 = -0.51$ and $q_2 = -0.01$. Note that the total charge on the trinitrogen anion is not exactly $-1$, but a tiny donation from the nanotube to the confined molecule has occurred. The results for this charge distributions are not listed because for both cases considered above, the resulting interaction energies are constantly lower (i.e. more favorable) by about 4 to 6 kcal/mol.

\bibliographystyle{unsrt}
\bibliography{library}

\end{document}
