\documentclass{article}

\usepackage{preamble}

\begin{document}

The Improved Lennard-Jones (ILJ) potential is given by
%
\begin{equation}
    V_{ILJ}(r) = \epsilon \left[ \frac{m}{n(r)-m} \cdot \Big( \frac{r_0}{r} \Big)^{n(r)} -
                                 \frac{n(r)}{n(r)-m} \cdot \Big( \frac{r_0}{r} \Big)^{m} \right] \, ,
\end{equation}
%
where
%
\begin{equation}
    n(r) = \beta + 4.0 \cdot \Big( \frac{r}{r_0} \Big)^2
\end{equation}
%
and $r = R_{i,j}$ is the distance between carbon atom $i$ and nitrogen atom $j$. The parameters $r_0$, $\epsilon$ and $\beta$ are set according to the atoms interacting, while $m$ is set according the to the partial charges of the interacting species.
The induction potential term for a $\text{N}_3^-$ anion interacting with carbon atoms is given by (in meV)
%
\begin{equation}
    V_{ind}(R_{i,1},R_{i,2},R_{i,3}) = - 7200 \cdot \alpha_c \cdot \frac{n_2(R_{i,2})}{n_2(R_{i,2}) - 4}
                        \left[ \frac{q_1}{R_{i,1}^2} + \frac{q_2}{R_{i,2}^2} + \frac{q_3}{R_{i,3}^2} \right]^2 \, ,
\end{equation}
%
where $R_{i,1}$, $R_{i,2}$ and $R_{i,3}$ are the distances between the carbon atom $i$ and the three nitrogen atoms, $q_1$, $q_2$ and $q_3$ are the partial charges on the nitrogen atoms, $\alpha_c$ is the polarizability of the carbon atoms and $n_2(R_{i,2})$ is equal to $n(r)$ using $r_0$ according to the interaction between $C-N_2$.\\
The interaction energy between a trinitrogen anion confined inside a carbon nanotube is given by
%
\begin{align}
    E_{int} = \sum_{i=1}^{N_C} V_{ILJ}^{C-N_{1,3}}(R_{i,1}) + V_{ILJ}^{C-N_{2}}(R_{i,2}) V_{ILJ}^{C-N_{1,3}}(R_{i,3}) + V_{ind}(R_{i,1},R_{i,2},R_{i,3}) \, ,
\end{align}
%
where $N_C$ corresponds to the total number of carbon atoms, the ILJ potentials $V_{ILJ}^{C-N_{1,3}}$ and $V_{ILJ}^{C-N_{2}}$ are defined according to the parameters given in \Cref{tab:ILJ_parms} and the polarizability $\alpha_c$ appearing in $V_{ind}$ is set to $1.2$, the partial charges as $q_1 = q_3 = -0.56$ and $q_2 = 0.12$ and finally the term $\tfrac{n_2}{n_2 - 4}$ is simplified to $1.0$ for the time being.
%
\begin{table}[h!]
    \centering
    \begin{tabular}{l|cccc}
        atom types      & $\epsilon$  & $r_0$     & $\beta$ & $m$ \\
        \midrule
        C-\ce{N_{1,3}}  & 5.205       & 3.994     & 6--9    & 6.0 \\
        C-\ce{N_{2}}    & 3.536       & 3.828     & 6--9    & 6.0 \\
    \end{tabular}
    \caption{ILJ parameters for the interaction of $\text{N}_3^-$ confined inside a carbon nanotube. Different values of $\beta$ are tested.}
    \label{tab:ILJ_parms}
\end{table}
%
\Cref{tab:eint_geo_1_ind_1} shows the interaction energies computed using the above formula and different values of $\beta$ for a representative geometry.
%
\begin{table}
    \centering
    \begin{tabular}{l|rrrr}
        $E_{int}$            & $\beta = 6.0$ & $\beta = 7.0$ & $\beta = 8.0$ & $\beta = 9.0$ \\
        \midrule
        $V_{ILJ}$ [meV]      &  -569.17      &  -517.63      &  -462.66      &  -403.66      \\
        $V_{ILJ}$ [kcal/mol] &   -13.12      &   -11.94      &   -10.67      &    -9.31      \\
        $V_{ind}$ [meV]      & -2493.52      & -2493.52      & -2493.52      & -2493.52      \\
        $V_{ind}$ [kcal/mol] &   -57.50      &   -57.50      &   -57.50      &   -57.50      \\
        \midrule
        $E_{int}$ [meV]      & -3062.68      & -3011.15      & -2956.18      & -2897.18      \\
        $E_{int}$ [kcal/mol] &   -70.63      &   -69.44      &   -68.17      &   -66.81
    \end{tabular}
    \caption{Interaction energies computed using ILJ and the induction term for different $\beta$ values. The term $\tfrac{n_2}{n_2 - 4}$ in the induction potential was kept fixed equal one. The reference MP2 value is -32.31 kcal/mol (-1401.13 meV).}
    \label{tab:eint_geo_1_ind_1}
\end{table}
%
A second series of test has been done by resetting the term $\tfrac{n_2}{n_2 - 4}$, which depends on $\beta$ and therefore will change accordingly.
\Cref{tab:eint_geo_1_ind_beta} shows the results in this case.
%
\begin{table}
    \centering
    \begin{tabular}{l|rrrr}
        $E_{int}$            & $\beta = 6.0$ & $\beta = 7.0$ & $\beta = 8.0$ & $\beta = 9.0$ \\
        \midrule
        $V_{ILJ}$ [meV]      &  -569.17      &  -517.63      &  -462.66      &  -403.66      \\
        $V_{ILJ}$ [kcal/mol] &   -13.12      &   -11.94      &   -10.67      &    -9.31      \\
        $V_{ind}$ [meV]      & -4033.24      & -3818.14      & -3657.00      & -3531.54      \\
        $V_{ind}$ [kcal/mol] &   -93.01      &   -88.05      &   -84.33      &   -81.44      \\
        \midrule
        $E_{int}$ [meV]      & -4602.40      & -4335.77      & -4119.67      & -3935.20      \\
        $E_{int}$ [kcal/mol] &  -106.13      &   -99.98      &   -95.00      &   -90.75
    \end{tabular}
    \caption{Interaction energies computed using ILJ and the induction term for different $\beta$ values. The term $\tfrac{n_2}{n_2 - 4}$ in the induction potential was let free to vary. The reference MP2 value is -32.31 kcal/mol (-1401.13 meV).}
    \label{tab:eint_geo_1_ind_beta}
\end{table}
%
A third series of calculations has been performed using NBO charges calculated on the MP2 density; namely $q_1 = q_3 = -0.51$ and $q_2 = -0.01$. Note that the total charge on the trinitrogen anion is not exactly $-1$, but a tiny donation from the nanotube to the confined molecule has occurred. The results for this charge distributions are not listed because for both cases considered above, the resulting interaction energies are constantly lower (i.e. more favorable) by about 4 to 6 kcal/mol.\\
\\
Maybe we should consider the partial charges also on the nanotube? In particular, the carbon atoms on the edges passivated by the hydrogen atoms show a large polarization, the carbons are charged by $-0.21$ and the hydrogens by $0.23$.

\end{document}
